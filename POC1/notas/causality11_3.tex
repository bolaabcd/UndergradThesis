No nodes in $Z$ can be a descendent of $X$ when we're using the back-door criterion... Pearl argues here why this is so.

He talks about strong ignorability.

He shows an alternative proof of the back-door criterion (simpler than the one he showed initially).

Pearl reminds of the methods of chapter two of finding all causal structures that are compatible with some given set of statistical independencies, assuming they're stable.

A set is \textit{c-admissible} when we can estimate the $do$ effect of $X$ on $Y$ by conditioning on this set.

\textit{c-equivalence} of two sets is when if we condition on one set or another, we obtain the same estimate.

Testing for \textit{c-equivalence} is a way of checking which model is better...

Pearl reinforces that strong ignorability is an opaque thing, not easily understood.


