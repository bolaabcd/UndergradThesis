Pearl says why we don't need to distinguish pre or post-intervention observations: we kind of can capture the effect of action plans by looking at some interventions and observing the rest... I need to review section 4.4... I didn't really get it...

Reminder: if we have linear models, everything becomes simpler.

Then he proceeds to answer questions about the do operator: It's necessary to model all changes so we get the desired result ($do(x)$ should change only $X$, nothing else).

One big advantage of the $do$ operator is to allow non-parametric estimations.

Maybe the axioms are not simple enough? I mean, in comparison with axioms in other areas...


