To me, this seems a bit ad hoc, again. But not that much actually... Now I think that the main thing is representing a way to reason with only the observables, without relying on how exactly the unobservables affect everything.

We say that $X$ \textbf{causally sustains} $Y$ relative to contingences in $W$ (which is a set of variables) if under some circunstances $u$, $X=x$, $Y=y$, no matter what we do to the variables in $W$, as long as $X=x$ we'll get $Y=y$, and there is a value $w'$ and $x'\neq x$ such that if we set $X=x'$ and the values of $W$ to $w'$, then $Y=y'$.

A consequence of this definition is that if $X$ causally sustains $Y$ relative to $W$, then there is a value $w'$ such that $X=x$ is necessary and sufficient for $Y=y$.

Basically Pearl is avoiding the consideration of alternative unobservables $u'$ by using interventions.

$W$ should not include all mediators between $X$ and $Y$, and Pearl in the next section will define rules of what variables should be included.
