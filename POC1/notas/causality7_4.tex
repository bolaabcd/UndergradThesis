Pearl compares his approach to one based on distances of worlds that satisfy some assumptions. He shows that both approaches are identical if the sysstem is a DAG (recursive).

Pearl mentions that conditioning on $do(x)$ is equivalent to transfering the mass from states with $X\neq x$ to states with $X=x$ according to some distance notion that was mentioned before, but this is the reason why he did it.

He also compares what he showed with the Neyman-Rubin Framework, and other previous work I don't know about (error-based definitions, instrumental variables, etc.), so I'm not going to focus on these parts.


