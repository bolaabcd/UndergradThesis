Pearl argued that the structural changes (interventions) are what really convey the meaning of actual cause that we want to convey intuitively.

\q{Sustenance} means \textit{sustaining} the effect in face of structural changes, and under some structural change $W=w$ be so that things could be different if we changed the cause (if no structural change is required, then it's an actual cause, otherwise it's a contributory cause).

Pearl finishes by sayng that it seems that what kind of causal notion to use depends on the goal. If the goal is to answer \q{how could this have been prevented?} then we might use the cause of necessity?. If the goal is \q{what was responsible} we want actual cause/susteinance (?), \q{how could we guarantee this} then we use sufficient cause and for \q{how to control when this happens} we use necessary and sufficient causation...

\subsubsection{REFORMULATION}

It seems that there was a problem with the notion of actual causation: if I understood correctly, we could have problems for contributing causes that are not direct parents of the effect...
