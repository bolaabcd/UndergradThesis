This looks more like a philosophy/curiosity chapter, so I'll not try to understand everything here completely.

Pearl is basically talking here about how strange would it be if something satisfied our definitions of causation but went in the reverse direction in time (something now causing something in the past, or something in the future causing something now).

He defines \textbf{statistical time} as any topological ordering of the variables according to at least one minimal causal structure consistent with the empirical distribution.

From what I understood, a Markov Chain has one statistical model that always goes foward, one that always goes backward, and others that given a fixed node (I believe we can fix a day for instance), we go foward and backwards from that node.

He says something about two coupled Markov Chains having only one possible time, conjectures that usually statistical time should coincide with physical time, and someone related this to the second law of the thermodynamics, but Pearl says it might be something else as the example does not follow that law (?) I didn't get this completely.

Also, as I understood it, it seems like we can change the direction of this unique statistical time sometimes by simply changing the coordinate system... He then argues that it's more a matter of how we prefer to represent things than something about the nature of reality. He also speculates that this kind of reasoning might have been naturally selected to give more value to finding out what will happen in the future given present information than what happened in the past that explains the present...

By quicly googling it, it seems like Pearl is a physicist (curiosity because he presented some Physics examples here, that at least to me are a bit obscure).
