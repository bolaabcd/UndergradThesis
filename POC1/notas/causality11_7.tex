Pearl talks a bit a bout linearity assumptions in counterfactual estimations. He points out that it's, in general, not possible to estimate $P(Y_{x}|y',x')$ even with infinite data samples.

In linear models, according to Pearl, it turns out that it's possible to fully estimate many types of counterfactuals... He proves some things related to this, I'm going to skip all.

It seems like counterfactuals are not $P(Y|do(X),Z)$: it seems like pearl considers that for this to be possible we need $Y\cup Z = \emptyset$ and $Y \cup X  \emptyset$, maybe $X\cup Z = \emptyset$... But for counterfactuals we can do this kind of thing... See the subsection here.

There is also a quick question on the generalization of twin networks method for more than two possible worlds.

Reminder: the twin networks method is a way to not store all background variables and simplify computations.

\subsection{\textbf{ANSWER TO ONE OF MY QUESTIONS!}}

$P(Y|do(x),z)$ is the probability of doing $X=x$ \textit{THEN}, \underline{\textit{\textbf{LATER}}}, observing $Z=z$!!!!!

$P(Y_x|z)$ is the probability of first seeing then doing.

Pearl says the two are equal if $Z$ is not a descendent of $X$ in the model. And he reinforces this is for a static model...
