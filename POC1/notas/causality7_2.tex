Pearl shows an economy example and argues why counterfactual questions are important.

Counterfactuals of the form \q{If $X$ were equal to $x'$ instead of $x$ in the past, than $Y$ would be equal to $y$} can be interpreted also as what \textbf{will happen} in the future with the value of $Y$, if we keep all other variables the same as in the past conditions and then set $X$ to $x'$.

Pearl mentions that counterfactual questions \textbf{are only useful if} the $f_i$ persist through time... For instance \q{I would be rich if I selected the right lottery numbers last week!} is useless, as in the future the conditions that determine the future results will change (at least according to the kind of information I have, the laws of physics will not change, and neither will whatever randomization algorithm they use).

Actually, he argues that they are \textbf{also useful if} we want to explain how/why something happenned.

He mentions that quantum phenomena (affected by the uncertainty principle) are situations in which counterfactuals might lose much of their utility, as the unobservables are never ever kept constant and we never know exactly when they'll repeat...

Pearl says also \textbf{why we can't use causal bayesian networks or $f_i$ as stochastic functions} here! He says that it entails the assumption that unities are homogeneous in the population, which might be appropriate in quantum phenomena but isn't in general, according to him.

\subsection{About explanations}

Pearl mentions that there are problems with notions such as the likelyhood ratio to explay what best explains some result that happenned. He says that by using causality concepts we can better capture the meaning of what we mean by \q{explanation}...

\textbf{General causes} are of the form $P(y|do(x))$ (in general, drinking poison is deadly), and \textbf{individual causes} are of the form $P(Y_{x'} = y'|x,y)$ (did Socrates die because of drinking poison?), the conditional part is supposed to filter results to the individual level.

Pearl argues that causal structures allow us to specify nuances in many causal questions.

Changes are called local in the space of mechanism (because different representations can have different notions of \q{local}, as with fourier transforms), so if we change something it will change only some local mechanisms...

He mentions a method of finding possible causal ordering of the variables from the equations that govern them, based on which variables depeend on which...
