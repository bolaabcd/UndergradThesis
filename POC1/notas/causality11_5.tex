Here he talks more about linear models. I'm going to mostly skip this.

The linear coefficients relate to unit interventional changes, not simply unity changes. 

The indirect effect of $X$ on $Y$ through $Z$ is what happens to $Z$ when we hold $X$ constant and set $Z$ to whatever value it would be if we acted on $X$.

Pearl provides an imaginary discussion in which someone tries to convince their PHD thesis \q{evaluators} why causality makes sense, and what exactly the causal claims and conclusions do mean.

Pearl refutes some other approaches and questions again, I'm going to mostly skip this too.

Pearl talks about the claim that if we have assymetrical relationships, then the $do$ operator could change the value of any combination of them, and this applies if we don't know the causal direction or if we simply don't have one.

Pearl kind of says that we can in fact model feedback loops and assume they reach equilibrium quickly if we want, but we'll still have one equation determining the value of each and every variable.

Reminder: Pearl calls systems with feedback \q{nonrecursive}.

\subsection{About causal ordering}

Pearl reminds here of the case in which we have a constraint in the system but not the directions of the arrows, and want to find them to determine a recursive system...

We have a set of constraints (a set of equations). If we have a set of actions that can modify one equation then the dependent variable of that equation is the variable that accounts for changes in the solutions of the entire system (?), if it's the same no matter which action we take.

I think it's nevertheless very complicated... At least I didn't get it completely.
