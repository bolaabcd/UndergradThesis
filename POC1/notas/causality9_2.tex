The probability of sufficiency is that chance of $Y$ not happening if $X$ had happenned given that neither really happenned.

The probability of necessity and sufficiency is the chance of $X$ triggering $Y$ and not $X$ turning $Y$ off.

Pearl also defines the Probability of Disablement and Probability of Enablement and explains in which situation they should be used. But to me these many metrics seems rather arbitrary... 

There is a (simple) relation between them and the probability of necessity and sufficiency, which might help understand them.

The probability of necessity doesn't change if we introduce extra inhibitors of the effect, and the probability of sufficiency doesn't change if we introduce extra alternative causes to $Y$.

Pearl defines $X$ as exogenous in relation to $Y$ as independence between the value of $X$ and the values $Y$ would take if $X$ is on or off. I find this strange, but maybe this just means that $X$ need to be a node without any arrow entering it? It's not exactly it...

I think a sufficient condition for $X$ to be exogenous in relation to $Y$ is for no backdoor path to exist between $X$ and $Y$. As if such existed, knowing the actual value of $X$ might help discover the value of a confounder (or error term depending on which part of the backdoor criterion was violated). Then, $do(x)$ is different from $do(x)|x$ because this extra information can help. Pearl mentions something like this, that there's a relation to the backdoor criterion.

Well, why couldn't we do some conditioning to solve the case where there's a backdoor but we can control it somehow?

Anyway, under this exogeneity assumption, we can find bounds on the probability of necessity and sufficiency. And other causal probabilities.

Another usefull condition is monotonicity, $Y$ is monotonic in relation to $X$ if $Y_x(u)$ is monotonic in $x$ for all $u$. In other symbols, if $y_x' \land y_{x'} = false$, if we cant both have $x$ on making $y$ off and $x$ off making $y$ on.

Under monotonicity and exogeneity we can find exactly the $PNS$ (Prob. of Necessity and Sufficiency) and others.

Under monotonicity and non-exogeneity we have other ways of identifying $PNS,PN$ and $PS$.


