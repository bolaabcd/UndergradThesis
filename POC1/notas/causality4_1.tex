Reactive vs deliberative interpretation of actions. Pearl calls the first one \q{act} (you did $Y$ because $X$ happenned to you) and the second one \q{action} (I did $Y$ because I thought about what would happen in some scenario, and decided $Y$ was the best decision). Actions turn into acts once executed, and what we measure and try to explain are the acts of agents present in our model.

Acts are kind of consequences of other things, actions are kind of free will-ish things.

One of the main advantages of causal modeling is to be able to predict the effect of complex actions from simpler parts, without the need to specify, for instance, "do X and Y" separately from "do X" and "do Y"...


