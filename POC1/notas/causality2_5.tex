Assuming that the distribution is stable, then there's an unique minimal causal structure up to d-separation.

\textit{patterns} are the causal structures with undirected edges whenever some minimal causal structure has one direction and other minimal structure has the other direction for that edge.

The algorithm that builds the pattern given a stable distribution consists simply of creating the undirected graph with an edge iff no conditioning makes the variable independent, then add arrows as mediators iff a and b are not adjacent and c is a common neighbour that is not in the set that separates them. After that, orient any edges that if oriented differently would make a cycle or a new  v-structure (remembering, v-structures are mediator structures).

Basically, it seems to me we're finding the undirected graph, directing it whenever we can see a collider, then directing the edges that are always directed that way if we don't have a directed cycle and we don't have unindentified colliders.

In summary, I think we can see the colliders only, we assume there's no cycle, and we can't differentiate between confounders and mediators.

Also, most importantly, this assumes we have the exact distribution, not an empirical observation of it.

Pearl mentions many ways of doing what the algorithm requires (how to do some parts is unespecified). There are fast solutions for linear gaussian simplifications.

I'll not note the details here, but some ideas are given in the book of how to imeplement the unspecified steps of the algorithm.

Pearl says that latent structures require special treatment, which will be covered a bit more in the next chapter. So, I guess all of this was for graphs without latent variables? Maybe he was building just the \q{causal Bayesian Network} model... \textbf{It might be important to understand this better}.
