\documentclass[oneside]{book}
\usepackage[utf8]{inputenc}
\usepackage{amsfonts}
\usepackage{geometry}
\usepackage{fancyhdr}
\usepackage{amsfonts,amsmath,amsthm,amssymb}
\usepackage{graphicx}
\usepackage{float}
\usepackage{hyperref}
\usepackage{xcolor}
\usepackage{lipsum}
\usepackage[mathscr]{euscript}
\usepackage{fancyvrb,fvextra}
\usepackage{svg}
\usepackage{soul}

\newcommand{\todo}[1]
{\marginpar{\baselineskip0ex\rule{2,5cm}{0.5pt}\\[0ex]{\tiny\textsf{#1}}}}

\newcommand{\R}{\mathbb{R}} % Real numbers
\newcommand{\Z}{\mathbb{Z}} % Integers
\newcommand{\N}{\mathbb{N}} % Naturals
\newcommand{\q}[1]{``#1"}
\newcommand{\D}{\mathbb{D}}
\newcommand{\DX}{\mathbb{D}(\mathcal{X})}
\newcommand{\DDX}{\mathbb{D}^2(\mathcal{X})}

\newtheorem*{medpondexist}{Existência de medianas ponderadas}
\newtheorem*{medehmin}{Sse mediana ponderada, então mínimo}

\newtheorem{theorem}{Theorem}[section]
\newtheorem{corollary}{Corollary}[theorem]
\newtheorem{lemma}[theorem]{Lemma}
\newtheorem{defin}{Definition}[section]

\newtheorem{remark}{Remark}[section]

\title{Notes POC1}
\author{Artur Gaspar}

\begin{document}

\maketitle
\tableofcontents

\chapter{Causality Chapter 1}

\section{Section 1.1: Introduction and review}

\textbf{Odds} are the fraction of probabilities. \textbf{Prior (predictive/prospective) odds} is $\frac{p(H)}{p(\neg H)}$, and the \textbf{Posterior (diagnostic/retrospective) odds} is $\frac{p(H|e)}{p(\neg H|e)}$. This is how much more likely the hypothesis is to be true than false a priori and after observing the event $e$. 

The \textbf{Likelyhood Ratio (Risk Ratio for epidemology)} is $\frac{p(e|H)}{p(e|\neg H)}$, remembering that Likelyhood is a function of $B$ in $p(A|B)$, while the probability is a function of $A$.

The formula is: Posterior Odds = Prior Odds $\times$ Likelyhood Ratio.

My interpretation of $p(H|e)$ is the probability we give to $H$ in the world where $e$ happens, thus if we do $\frac{p(H|e)}{\neg H|e}$ we're seeing how more probable (multiplicatively) $H$ is to be true in this world, and if we do $\frac{p(e|H)}{p(e|\neg H)}$ we're seeing how much more likely is the event $e$ to happen in the world in which $H$ is true than in the world in which it's not (it's a comparison accros worlds).

I interpret the likelyhood ratio as how many more times the evidence appears in the world where $H$ is true than in the world where it's not.

So how more likely the hypothesis is to be true than false, after we observe the event = how more likely the hypothesis was to be true before the observation was made times $\times$ how many more times the evidence appears in the world where $H$ is true than in the world where it's not.

\textbf{Odds of hypothesis after $e$ = odds before $e$ $\times$ how much more $e$ happens in $H$ than in $\neg H$.}

Covariance is the expected value of $(X-E[X])(Y-E[Y])$, distance to the averages, $cov(X,X) = var(X) = (std(X))^2$, and $corr(X,Y) = \frac{cov(X,Y)}{std(X)std(Y)}$.

Regression coefficient when estimating $Y$ using $X$ is $corr(X,Y)\times \frac{std(Y)}{std(X)}$, which is how much $Y$ will change by unity of $X$ we change, if we use the line that minimizes the quadratic error of the $Y$ estimate. I kind of interpret this as $\frac{(\text{X-unities})}{(\text{X-unities per standard devition of X})} \times corr(X,Y) \times std(Y) = (\text{number of standard deviations of X}) \times corr(X,Y) \times std(Y) = (\text{number of standard deviations of Y}) \times (\text{Y-unities per standard deviations of Y}) = (\text{Y-unities})$. The strange thing with this interpreatation is that $corr(X,Y) = (\frac{\text{standard deviations of X}}{\text{standard devations of Y}}) = \frac{\text{standard deviations of Y}}{\text{standard deviations of X}}$, is the function that given one ammount of standard deviations returns the other one... Maybe this is a reflection of the limitations of the linearity assumption?

Finally, the graphoid axioms for independence of random variables (all of them conditioned on $Z$, and I simplified a little bit):

\begin{enumerate}
	\item Symmetry: $X$ is independent of $Y$ iff $Y$ is independent of $X$
	\item Decomposition, Weak Union and Contraction: $X$ is independent of $YW$ iff (($X$ is independent of $Y$) and ($X$ is independent of $W$ conditional on $Y$)).
	\item 
\end{enumerate}

Os axiomas são (todos condicionado para um Z qualquer, versão simplificada por mim):

1) Simetria: X independente de Y <=> Y independente de X

2) Decomposição, Weak Union, Contraction: X independente de YW se e somente se ((X independente de Y) e (X independente de W condicional a Y))

Isso tudo sse ((X independente de W) e (X independente de Y condicional a W)), só trocar o nome de Y e W, a YW = WY aqui...

Esse acima implica decomposição diretamente, implica Weak Union tbm diretamente, e implica contraction diretamente tbm. Os três juntos levam a esse bem direto tbm, são realmente equivalentes.

3) Intersection (só se não tem nada com chance 0, strictly positive distributions): X independente de W dado Y e X independente de Y dado W implica que X independente de WY. Acho que é um sse tbm pelos outros axiomas, mas ele fala que essa ida só é válida em distribuições estritamente positivas.

Em resumo: 1) independência é simétrica, 2) independente de duas coisas é o mesmo que independente de uma e independente da outra dado a primeira / saber y,w não ajuda na minha estimativa de x <=> y sozinho não ajuda e w não muda se eu já sei y, e 3) independente das duas é o mesmo que ser independente de uma dado a outra e da outra dado a uma (vale só pra distribuições estritamente positivas aparentemente).


NOTAS NO PAPEL PASSAR PRO PC


\section{Section 1.2: Bayesian Networks}

\subsection{1.2.1) Conventions}

\textbf{skeleton} of a graph is the undirected version of it.

In this book, a \textbf{path} might not follow the direction of the edges.

\textbf{Family} of a graph is a node and it's parents.

\textbf{Root} is a node without parents and \textbf{sink} a node without children.

\textbf{Tree} is a connected graph with at most one parent per node (one node can point to many but only one node can point to it), and \textbf{chain} is one with at most one child per node (onde node can point to only another one, but many can point to it).


\subsection{1.2.2) Bayesian Networks}

One of the main goals is to represent an joint distribution with less data, which is possible if every variable is independent of almost all others.

The \textbf{Markovian parents} of a node is a minimal set of nodes that, conditioned on them, the value of the node is independent from the value of all other nodes. It's a set of variables that we can condition on to ignore the rest when estimating the initial node, but such that we can't remove any variable from this set.

This set is unique if the joint distribution is strictly positive, and this implies an unique Bayesian Network. 


\textit{I believe that if it's not strictly positive, then if we consider that the parents must be minimal, it might be impossible to draw a Bayesian Network, otherwise we accept non-minimal sets of parents and acknoledge taht we might have more than one BN.} See the subsection \q{Instability of parents for non-positive distributions}.


We say that $G$ represents $P$, or $G$ is compatible with $P$ if we can decompose $P$ with the information we extract from $G$ (the DAG). For instance, $p(a1,b1,c1,d1,e1) = p(b1|a1)p(c1|a1)p(d1|b1,c1)p(e1|d1)$ is the decomposition for the graph with $A\rightarrow B$, $A\rightarrow C$, $B\rightarrow D$, $C \rightarrow D$ and $D\rightarrow E$.


\subsection{Instability of parents for non-positive distributions}

I think that it's possible to have more than one minimal set (of Markovian Parents) if the third graphoid axiom is not satisfied, because then we can have $X$ independent of $Y$ given $Z$ and of $Z$ given $Y$, but not on $YZ$. So we might want to require the distributions to be strictly positive...

Take for instance the following joint for $A,B,C$ binary:

\begin{enumerate}
    \item $p(a1,b1,c2) = \frac{1}{2}$.
    \item $p(a2,b2,c1) = \frac{1}{2}$.
    \item All other probabilities equal $0$.
\end{enumerate}

Here if we know one value we know the other two, so $\{B\}$ or $\{C\}$ are minimal markovial parents for $A$; $\{A\}$ or $\{B\}$ are minimal for $C$; and $\{A\}$ or $\{C\}$ are minimal for $B$. So, we kind of can't create an undirected graph that represents the dependencies well... One node will connect to other two, but it acually depends on only one (any one)...


\subsection{1.2.3) d-separation}

This is a criterion to extract the conditional independences between variables from the graph.

$X$, $Y$ and $Z$ here can be sets of more than one variable.

We say that a \textit{path} is \textbf{$d$-separated} or \textbf{blocked} if either $Z$ has a variable in the middle of the way or as a confounder, or the path has a collider which is not in $Z$ and no descendent of the collider is in $Z$. 

We say that $Z$ $d$-separates $X$ from  $Y$ if it does so for every path from $X$ to $Y$.

\textit{It's really important to consider the descendent part! Conditioning on a variable unblocks every collider that is reachable in reverse order (following the arrows reversed) from this variable.}


\textbf{$X$ and $Y$ are $d$-separated by $Z$ if and only if for all distributions compatible with the independencies of $G$, $X$ and $Y$ are conditionally independent given $Z$. Also, if they are not $d$-separated, almost all distributions make they dependent (they don't say \q{how much indepedent}).}


\textbf{Selection bias}, \textbf{Berskon's paradorx} or \textbf{explaining away effect} is the situation in which after conditioning on one variable we render two others dependent (knowing that one does not have a specific value lets us increase the chance of another, for instance).

\textbf{Observational Equivalence} is the situation in which we have two graphs such that any distribution compatible with one is also compatible with the other.

It happens iff they have the same undirected structure and the same \q{$v$-structures}, which are converging arrows without a connection between their tails: $X \rightarrow Y \leftarrow Z$ but no arrow between $X$ and $Z$ forms a $v$-structure.


\subsection{1.2.4) Inference with BNs}

The book comments a bit on how we could try to estimate conditional probabilities of some variables given the observation of others. I'm not going to focus on this.



\section{Section 1.3: BNs with causal directions}

A Causal Bayesian Network is a Bayesian Network with causal directions.

We say that a distribution after an intervention is compatible with the CBN if it's Markov relative to it (we can decompose the joint with respect to the BN, or the parents make the childs independent of non-descendents), the chance of the interventions happening is one, \textit{and the conditional probabilities remain the same for variables we didin't act on}.

The joint after the intervention can be factorized as $P(v) = \Pi_{i|V_i \notin X} P(v_i|pa_i)$, which is basically the original joint without the $P(v_i|pa_i)$ of the variables we acted on. $v$ is a vector here (the entrances are the values of the random variables that are represented by the nodes of the CBN).

Two properties: $P_{pa_i}(v_i) = P(v_i|pa_i) = $ interventions are according to the conditionals, and $P_{pa_i,s}(v_i) = P_{pa_i}(v_i) = $ no interventions besides the one in the parents can influence a variable

Pearl argues that the advantage of causal models is to transport results to other environments and predict the results of changes that aren't purely observacional.


\section{Section 1.4}

\subsection{Laplacian vs Stochastic model}

The Laplacian one has deterministic functions and unobserverd probablistic variables, he stochastic one is more similar the Bayesian Network approach, if I understood correctly

Pearl says that this is more general than probablistic functions, but to me this just makes sense if by stochastic he doesn't mean something like a Markov Chain instead of the function, as this would certanly be more general... The BNs do not really have Markov Chains, but conditional probabilities, maybe that's what he means?


\subsection{1.4.1: Structural Equations}

\textbf{Structured Equation Models} are defined defining each variable as a function of the parents and unobserved variables (erros). If it's linear, then its a \textbf{Linear Structured Equation Model}.

In the linear models, the coefficients are the variation rates per forced variation of a value, in the sense that it's how much the value would change if we changed only that value by one unity.

It's usually assumed that the error terms are independent, if they are dependent we represent a dotted double-headed arrow between the variables involved.

The hyerarchy of Causal problems defined by Pearl are:

\begin{enumerate}
    \item \textbf{Predictions} are the \q{what if we found out that the value of this other variable was this?}
    \item \textbf{Interventions} are the \q{what if we set the value of this other variable to this?}
    \item \textbf{Counterfactuals} are the \q{what would be the value of this variable if the value of this other one was that instead of this?}
\end{enumerate}

\subsection{Definitions and equivalences between SCMs and BNs}

\textbf{Causal Diagram} is the diagram obtained by connecting the parents to the childs according to the structural equations. If this graph is a DAG, then it's \textbf{semi-Markovian}, and if the erros are independent, then it's \textbf{Markovian}. If it's semi-markovian, the joint is completely determined by the distribution on errors.

\textit{If the model is markovian, then this is a valid Causal Diagram: given the parents, a node is idependent of all other non-descendants.} The proof is just to get the full graph, with the errors, then notice that we can remove the errors without losing independencies.

Pearl says that this is implied if we include every variable that might be a causa of two or more others, and that there is no correlation without causation...

The idea seems to look at the data and determine all probabilities first, even without knowing the deterministic functions (and the errors or distribution on errors) themselves... For any joint distribution compatible with a bayesian network, there is always at least one Functional Model with this same network (and Pearl mentions that usually there are infinitely many) that generates it with some values for the error/unobserved variables.

So, I think this is what he meant before, that the functional models are more general: we can encode in them anything we could encode in a BN.

\subsection{1.4.2: Probabilistic Predictions}

Four advantages mentioned by Pearl of using the graphical representation of Causal Models are:

\begin{enumerate}
    \item The conditional independencies do not depend on the specific functions themselves, so if we can represent something in the causal model even with limited information, and given the model we don't need to compute anything to know whether some variables are independent given others.
    \item It's simpler to specify the connections, and the model has few parameters.
    \item It's simpler to think of whether or not the parent set has all relevant variables that are a direct cause of some variable, instead of checking whether they make this variable independent of the others when we condition on them (and are a maximum set that does that).
    \item If something changes, the change might be local on some variables only, and with these models we can model this change by changing less the model, instead of recomputing everything from scratch.
\end{enumerate}


\subsection{1.4.3: Interventions}

\subsection{1.4.4: Counterfactuals}

\subsection{Notes}

I still am a bit confused about being able to have more than one set of parents per node if the distribution is not strictly positive... What do we do about that? What if there is a logical limitation, and an example that's better (and harder to find the problem) than just two equal variables causing another? Would everything break or is it stable to lead to an \q{almost zero} probability when it would be zero?


\end{document}
