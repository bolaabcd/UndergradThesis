\documentclass{article}
\usepackage{amsmath}
\usepackage{amssymb}
\usepackage{amsthm}
\usepackage{amscd}
\usepackage{amsfonts}
\usepackage{dsfont}
\usepackage{fancyhdr}
\usepackage{enumerate}
\usepackage{steinmetz}
\usepackage[utf8]{inputenc}
\usepackage[margin=1in]{geometry}
\usepackage{color}
\usepackage{hyperref}

\title{Relações entre Causalidade, Justiça, Privacidade e Acurácia (+Explicabilidade?) em Aprendizado de Máquina (talvez mais foco em privacidade vs fairness porque as vezes os erros de privacidade fariam ficar mais justo?)\large\\ Tipo de Pesquisa: Científica\\Orientador: Mário Sérgio Alvim}
\author{Artur Gaspar}
\date{05/04/2024}


\begin{document}
\maketitle

\section{Introdução}

\textit{Qual o problema a ser resolvido ou questão a ser investigada no projeto? Por que ele é importante? Listar os objetivos gerais e específicos do trabalho.}

\textit{Na Introdução, ao descrever seus objetivos, o aluno deve informar o objetivo geral a ser atingido no final do POC II / MSI II, e depois listar os objetivos específicos de cada um dos semestres (POC/MSI I e POC/MSI II). Esses objetivos deverão ser revisitados no próximo semestre, quando for elaborada a proposta do POC/MSI II. A mesma coisa é válida para os resultados esperados. Ao fim de um ano, qual é o resultado esperado? Para alcançar esses resultados, o que precisa estar pronto ao final do POC/MSI I?}

No POC1, uma revisão dos conceitos de causalidade, Fairness, Privacidade e Acurácia em aprendizado de máquina (talvez explicabilidade?), e as relações entre essas coisas. No POC2, poderia ser mais sobre as relações entre essas coisas.

\begin{enumerate}
    \item Objetivos POC1: Revisar Causalidade, Fairness, Privacidade, Acurárcia (Explicabilidade? Teoria da Informação?)
    \item Objetivos POC2: Falar das relações entre todos esses conceitos, tudo que se sabe e talvez mais um pouco.
    \item Resultado esperado daqui a um ano é ter uma apresentação completa do que se tem de mais recente sobre os conflitos e sinergias entre esses fatores todos.
    \item Pra isso ao final do POC1 tem que ter uma revisão de métodos de fazer classificações mais justas, privadas e acuradas, e o que causalidade tem a ver com isso (talvez explicabilidade)
\end{enumerate}

\section{Referencial Teórico}

\textit{Apresentar os conceitos pertinentes para que o leitor entenda o problema e sua importância. Nesse momento, o aluno pode não ter domínio completo dos trabalhos relacionados. Fazer um breve resumo das soluções já existentes na literatura/mercado e como elas se comparam ao trabalho proposto. No caso de trabalhos de cunho tecnológico, listar ferramentas que resolvem o mesmo problema sendo tratado.}

\begin{enumerate}
\item Citar os artigos de métodos de fairness, Privacidade e Causalidade. 
\item Os papers discutidos na viagem pra Paris. Difeririam do nosso porque a gente quer colocar causalidade no meio, e talvez até melhorar um pouco os anteriores.
\end{enumerate}

\section{Metodologia}

\textit{Quais os principais passos previstos (com uma breve descrição) para execução do projeto? Como pretende-se abordar o problema?}
\begin{enumerate}
\item Ler artigo
\item Ler o livro de causalidade
\item Ler um pouco do resto do livro de Teoria da Informação
\item Tentar demonstrar alguma coisa, talvez
\end{enumerate}

\section{Resultados Esperados}

\textit{O que se pretende obter ao final do trabalho?}
\begin{enumerate}
\item Uma revisão da literatura
\item Desenvolver resultados teóricos novos sobre as relações de causalidade com fairness, privacidade com fairness, etc.
\end{enumerate}

\section{Etapas e Cronograma}

\textit{Descrever o cronograma previsto para a realização dos passos definidos na Metodologia com resolução em nível de semanas.}


\section{Referências}

\bibliography{poc1}

\end{document}
